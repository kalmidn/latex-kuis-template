\documentclass[12pt]{article}
\newenvironment{problem}[2][Problem]{\begin{trivlist}%Label "Problem" dapat diganti dengan custom label pada opsi []
		\item[\hskip \labelsep {\bfseries #1}\hskip \labelsep {\bfseries #2.}]}{\end{trivlist}}

%package yang digunakan
\usepackage{amssymb}
\usepackage{amsmath}
\usepackage{graphicx}
\usepackage{wrapfig}
\usepackage{amssymb}
\usepackage{caption} %untuk modifikasi caption pada gambar dan tabel
\graphicspath{ {./images/} }
\usepackage{blindtext} %%%%%package tambahan khusus untuk template ini. Saat penggunaan untuk pembuatan dokumen asli bisa di comment

\begin{document}
	\renewcommand{\labelenumi}{(\alph{enumi})} %mengganti enumerate angka dengan alphanumeric
	
	\title{Judul kuis/tugas}
	\author{NAMA MATA KULIAH\\ Dosen: XXZZYY}%Dapat ditambah baris
	\date{DD MM YYYY (HH.MM-HH.MM)}%Tgl tugas diberikan/deadline pengumpulan/pelaksanaan kuis
	\maketitle
	
	%%%%%% Peraturan Kuis/Tugas
	\begin{center}
		\fbox{\fbox{\parbox{5.5in}{\begin{itemize}
						\item Aturan 1
						\item Aturan 2
						\item Aturan 3
		\end{itemize}}}}
	\end{center}
	\pagebreak
	
	\begin{problem}[Soal]{1} 
		\blindtext
	\end{problem}
	
	\begin{problem}[Soal]{2}
		\blindtext
	\end{problem}
\end{document}
