\documentclass[12pt]{article}
\newenvironment{problem}[2][Problem]{\begin{trivlist}
		\item[\hskip \labelsep {\bfseries #1}\hskip \labelsep {\bfseries #2.}]}{\end{trivlist}}
\usepackage{amssymb}
\usepackage{amsmath}
\usepackage{graphicx}
\usepackage{wrapfig}
\usepackage{amssymb}
\usepackage{caption}
\graphicspath{ {./images/} }

\begin{document}
	\renewcommand{\labelenumi}{(\alph{enumi})}
	
	\title{Kuis 2}
	%\title{Problem Set Arus Listrik dan Rangkaian DC}
	\author{Fisika Matematika I Kelas C, 2022-2023\\ Dosen: Mohammad Haekal, Ph.D.}
	%\author{Dosen: Mohammad Haekal, Ph.D.}
	\date{24 November 2022 (15.30-17.10)}
	\maketitle
	
	
	\begin{center}
		\fbox{\fbox{\parbox{5.5in}{\begin{itemize}
						\item Kuis bersifat \textbf{\textit{closed-book}}. Tidak diperbolehkan ada kerjasama sesama peserta kuis dalam bentuk apapun.
						\item Pada saat kuis berlangsung, hanya boleh ada alat tulis, kertas soal, kertas jawaban dan lembar coretan di atas meja masing-masing peserta. Seluruh barang lain disimpan di dalam tas yang dikumpulkan di depan kelas sebelum kuis berlangsung.
						\item Beri nama dan NRP yang jelas untuk masing-masing lembar soal, lembar jawaban, \textbf{dan} lembar coretan.
						\item Penulisan jawaban boleh tidak sesuai urutan nomor. Namun penulisan nomor pada tiap jawaban wajib dicantumkan dengan jelas.
						%\item Masing-masing peserta wajib membawa kalkulator masing-masing dan tidak diperkenankan menggunakan ponsel sebagai kalkulator.
						\item  Setelah selesai mengerjakan kuis, kumpulkan \textbf{lembar soal, lembar jawaban dan lembar coretan} ke pengawas. 
						\item Segala indikasi bentuk kecurangan, akan diberikan sanksi berupa \textbf{nilai E} secara otomatis untuk mata kuliah ini.
		\end{itemize}}}}
	\end{center}
	\pagebreak
	
	\begin{problem}[Soal]{1}%Boas 3-4.23
		Buktikan apakah $\vec{B}=0$ jika diketahui bahwa $\vec{A}=2\ \hat{i}-3\ \hat{j}+k$ dan:
		\begin{enumerate}
			\item $\vec{A}\cdot\vec{B}=0$
			\item $\vec{A}\times\vec{B}=0$
			\item jika kedua nilai $\vec{A}\cdot\vec{B}=0$ dan $\vec{A}\times\vec{B}=0	$
			\end{enumerate}
	\end{problem}
	
	\begin{problem}[Soal]{2}%Boas 3-5.27
		Tinjau dua buah garis yang dinyatakan oleh
		\begin{equation*}
			\frac{x-1}{2}=\frac{y+3}{1}=\frac{z-4}{3}
		\end{equation*}
		dan
		\begin{equation*}
			\frac{x+3}{4}=\frac{y+4}{1}=\frac{8-z}{4}
		\end{equation*}
		dengan menggunakan metode vektor untuk menyatakan garis dan bidang, buktikanlah apakah kedua garis tersebut berpotongan di satu titik?
	\end{problem}
	
	\begin{problem}[Soal]{3}%SEssentials Linear Algebra
		Dengan menggunakan \textbf{metode reduksi baris}, carilah solusi dari sistem persamaan linier berikut:
		\begin{equation*}
			\begin{split}
				x_1-2x_2+2x_3-x_4 &= -14\\
				3x_1+2x_2-x_3+2x_4 &= 17\\
				2x_1+3x_2-x_3-x_4 &= 18\\
				-2x_1+5x_2-3x_3-3x_4 &= 26
			\end{split}
		\end{equation*}
	\end{problem}
	
	\begin{problem}[Soal]{4}%SEssentials Linear Algebra
		Dengan menggunakan \textbf{metode aturan Cramer}, carilah solusi dari sistem persamaan linier berikut:
		\begin{equation*}
			\begin{split}
				5x+2y+z &= 3\\
				2x-y+2z &= 7\\
				x+5y-z &= 6
			\end{split}
		\end{equation*}
	\end{problem}
	
%	\begin{problem}[(4 poin) Soal]{5}%Serway 5.73
%		Seorang pemodif membeli sebuah mobil bekas untuk diubah menjadi mobil balap. Setelah dilakukan analisis, dia menemukan bahwa mobil tersebut mampu berakselerasi sebesar $8.40\ km/(hr\cdot s)$ di trek lurus. Dengan melakukan simulasi sederhana, pemodif tersebut mengetahui bahwa dengan memodifikasi mesinnya, dia bisa meningkatkan gaya dorong horizontal total yang diterima oleh mobil sebesar $24.0\%$. Namun, dengan biaya yang lebih murah, dia juga bisa menghilangkan material dari bodi mobil yang dirasa tidak diperlukan pada sebuah balapan untuk menurunkan massa mobilnya sebesar $24.0\%$.
%		\begin{enumerate}
%			\item Pilihan mana di antara dua pilihan tersebut yang akan menghasilkan peningkatan lebih besar pada akselerasi mobil?
%			\item Jika pemodif tersebut melakukan kedua-duanya, berapa akselerasi yang bisa dicapai?
%		\end{enumerate}
%	\end{problem}
	
	
	
	%\includegraphics[]{}
\end{document}
